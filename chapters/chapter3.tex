\chapter{进一步阅读}
上一章介绍了基本使用方法,但是因为每个人论文的内容千差万别,形式也必然各式各样,所以为了拓展一下应用范围,本章的目的在于简要介绍一下如何利用 LaTeX 强大的功能(主要是宏包的扩展)来完成形形色色的排版任务。当然,这里的演示也只是提纲挈领式的。

\section{文档}
文档一般指技术领域的说明、帮助文件或者一些教程。阅读文档是学习技术的不二法门,你可能遇到的大多问题都可以在文档中找到答案。当然,运用搜索引擎也是一个好办法。还有就是可以向有经验的人提问,这都可以帮助解决问题。尽快学会如何去解决问题比解决某个问题更重要。

这里推荐一些学习的地方:
\begin{itemize}
	\item LaTeX 编辑部,\url{http://zzg34b.w3.c361.com}。提供了比较详尽的教程、宏包说明等文档集合,如果喜欢纸质版教程,《LaTeX2e 完全学习手册》就是一本不错的书。
	\item 中文 TeX 论坛,\url{bbs.ctex.org}。可以交流或寻求帮助。
\end{itemize}

\section{宏包}
宏包就是 Macro package 的字面翻译。关于 Macro(宏)这个计算机领域常见的概念,你不必深究,只要知道对 LaTeX 进行功能扩充就像搭积木似的调用 package,从而实现各式各样的功能\cite{latex2e}。调用宏包也十分容易,通常在源文件导言区即 \verb|\begin{document}| 之前使用 \verb|\usepackage{宏包名}| 调用即可,也可以直接在本文类 NJUbachelor.cls 文件加载宏包。

比如本文档导言区使用了 fancyvrb 这个宏包用来获得更灵活的抄录环境 \verb|Verbatim| 的功能。本文类默认加载了 ctex, booktabs, graphicx, longtable 等宏包,详见 NJUbachelor.cls 源代码。

\section{几个例子}
之前基本要素中给出的都是最常用的功能,但是每个人都有各式各样的需求,可以通过使用各种宏包来实现。这里简单给几个具体的例子以作演示,可以使你进一步地体会到使用 LaTeX 的方便之处。

\subsection{代码}
程序代码功能就像之前提到的,使用 fancyvrb 这个宏包扩展。效果请参见附录 \ref{app:src} 所示。

\subsection{化学式}
这个是某些专业同学需要使用的一个功能,调用 mhchem 宏包后就可以方便地使用。如:
\begin{Verbatim}[frame=single]
\begin{gather*}%这里使用无序号居中多行公式环境
\ce{BrO3- + HBrO2 + H+ ->[k_3] 2BrO2 + H2O}\\
\ce{BrO2 + Ce^{3+} + H+ ->[k_4] HBrO2 + Ce^{4+}}\\
\ce{2HBrO2 ->[k_5] BrO3- + HOBr + H+}
\end{gather*}
\end{Verbatim}
产生
\begin{gather*}%这里使用无序号居中多行公式环境
\ce{BrO3- + HBrO2 + H+ ->[k_3] 2BrO2 + H2O}\\
\ce{BrO2 + Ce^{3+} + H+ ->[k_4] HBrO2 + Ce^{4+}}\\
\ce{2HBrO2 ->[k_5] BrO3- + HOBr + H+}
\end{gather*}

\subsection{SI 单位}
这也是理科生常用的一个功能,可以通过 siunix 宏包取得方便扩展。
比如 
\begin{Verbatim}[frame=single]
$\mu_1=\SI[inter-unit-product=\ensuremath{{}\cdot{}}]
{2.09e-23}{\joule\per\tesla}$
\end{Verbatim}
产生 $\mu_1=\SI[inter-unit-product=\ensuremath{{}\cdot{}}]{2.09e-23}{\joule\per\tesla}$ 的效果。

举这三个例子,相信可以举一反三。其实宏包的具体使用方式在宏包的说明文档中都有,参考文档学习使用就可以了。
\section{最后的话}
到这里对本 LaTeX 模板使用的介绍已经告一段落。但解决问题的方法总有好多,会有很多方法去做同一件事。比如,如果你仍喜欢使用 Word 或 Pages 等文本处理软件,完全也可以先自己做一个模板出来然后再补充内容(这才是有效率的方式!),为此我推荐《Word 的排版艺术》这本书,相信也会有点帮助。

而如果是文科比如历史、中文等系科,其实还有更加“轻量级”的解决办法, 比如使用 Markdown 语言,然后利用 pandoc 转为 LaTeX 格式,这样写作起来就会更加舒畅。这里权当是抛砖引玉了,发挥你的想象力和才能,总会有好方法!
